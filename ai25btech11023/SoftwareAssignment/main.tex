\documentclass{article}
\usepackage{amsmath}
\usepackage{graphicx}
\usepackage{subcaption} 
\usepackage{multicol}
\title{Software Assignment: Image
Compression Using Truncated SVD}
\author{AI25BTECH11023 - Pratik R}
\date{}

\begin{document}

\maketitle
\let\vec\mathbf
\section{Summary}

The singular value decomposition of a matrix $A$ is the factorization of $A$ into the
product of three matrices $A = U\Sigma V^\top$ where the columns of $U$ and $V$ are orthonormal and the matrix $\Sigma$ is diagonal with positive real entries.
Let $A$ be some $m\times n$ matrix. We are interested in finding the basis vectors $\vec{v_i}$ of the row space of $A$ in $R^m$ such that after transformation gives scaled basis vectors $\vec{u_i}$ of the column space of $A$ in $R^n$, that is
\begin{align}
    A\vec{v_i}= \sigma _i \vec{u_i}
\end{align}
$\vec{v_i}$ are the columns of $V$ and $\vec{u_i}$ are the columns of $U$. $\sigma _i$ is the factor by which the transformed vector is scaled, $\sigma _i$ being the diagonals of matrix $\Sigma$. Adding Eq 1 for all i's upto r, we get $AV= U\Sigma$.

multipying by $V^\top$ on right
\begin{align}
    AVV^\top= U \Sigma V^\top 
\end{align}
Since $V$ is orthonormal matrix
\begin{align}
    A =  U \Sigma V^\top
\end{align}

In this video it is shown that by computing $A^\top A$ we can compute these singular values $\sigma _i$ and $\vec{v_i}$ in the following manner.   
\begin{align}
    A = U \Sigma V^\top 
\end{align}
\begin{align}
   A^\top A =   V\Sigma U^\top U \Sigma V^\top
\end{align}
\begin{align}
    A^\top A =   V\Sigma \Sigma V^\top = V \Sigma^2 V^\top
\end{align}
    This is a case of spectral decomposition in which $\Sigma^2$ contains eigenvalues of $A^\top A$ as diagonals. Matrix $V$ can be found by computing eigenvectors of $A^\top A$.For $U$ eliminate $V^\top V$ by taking $A A^\top$.
\section{Abstract}

The SVD is useful in many tasks. The data matrix $A$ is close to a matrix of low rank(k) and it is useful to find a low rank matrix which is a good approximation to the data matrix .From the singular value decomposition of $A$, we can get the matrix $A_k$ of rank k which best approximates $A$. This is what we will be using as the basis for image compression.


\section{Algorithm Selection}

There are several algorithms that can be used to calculate the eigenvalues of a matrix.
Some of these algorithms are listed below.

\begin{table}[h!]
    \centering
    \begin{tabular}{|c|p{6cm}|}
    \hline
    \textbf{Algorithm}        & \textbf{Description} \\
    \hline
    Power Iteration           & Iteratively finds the largest eigenvalue by repeatedly multiplying the matrix with a vector. \\
    \hline
    Inverse Iteration         & Finds the smallest eigenvalue by inverting the matrix and applying power iteration. \\
    \hline
    QR Algorithm              & Uses QR decomposition iteratively to compute all eigenvalues of a matrix. \\
    \hline
    Lanczos Algorithm         & Reduces a large sparse symmetric matrix to tridiagonal form for eigenvalue computation. \\
    \hline
    Arnoldi Iteration         & Generalizes the Lanczos method for non-symmetric matrices to compute a few eigenvalues. \\
    \hline
    Jacobi Method             & Rotates pairs of rows and columns to diagonalize the matrix iteratively. \\
    \hline
    Davidson Algorithm        & Specialized for large sparse matrices, builds a subspace to approximate eigenvalues. \\
    \hline
    \end{tabular}
    \caption{Eigenvalue Calculation Algorithms and Their Descriptions}
    \label{tab:eigenvalue_algorithms}
\end{table}
Nevertheless, for this project I have use Power Iteration method due to relatively 
code for algorithm can  be found in \em codes/c\_main/main.c \em
\section{Algorithm}
 \begin{enumerate}
     \item Read image using \em stbi\_load \em and storing it in 1D Array($m\times n$)
     \item multiply to get $A^\top A$
     \item Use Power iteration method to get most dominant eigenvector of $A^\top A$. Make sure to normalise it during each iteration.
     \item Corresponding eigenvalue($\lambda$) can be found by norm of  $A^\top A\vec{v}$ where $\vec{v}$ is the eigenvector
     \item To find the next eigenvector we remove the contribution of $\vec{v}$ from $A^\top A$, that is 
     \item $A^\top A\leftarrow A^\top A - \lambda v v^\top$ and repeat from step 3 to find the next dominant eigen vector
     \item These eigenvectors will form  basis of row space for the matrix $A$ and eigenvalues of $A^\top A$ will serve as squares of singular values. Basis of column space can be computed by $u = \frac{Av}{||Av||}=\frac{Av}{\sigma}$ 
     \item Take only best k eigenvalues to reconstruct the matrix $A_k$ via Truncated SVD.
     \item Reconstruct the matrix $A_k$ by the given equation $A_k = U_k \Sigma _k V_k^\top$ 
     \item In the code, $A_k$ is reconstructed by adding $\sigma _i \vec{u_i} \vec{v_i}^\top$ over i = 1 to k, that is
     \item $A = \sum_{i=1}^{k}\sigma_i\vec{u_i}\vec{v_i}^\top$  
     \item use \em stbi\_write\_jpg \em to write 1D Array back to jpg.
 \end{enumerate}

\textbf{Note}: iteration is set to over 200 and the reconstructed image for different values of k is shown in the next section.
\newpage
\section{Image output}
For Einstein.jpg
\begin{figure}[h]
    \centering
    \begin{subfigure}[b]{0.3\textwidth}
        \includegraphics[width=\textwidth]{figs/einstein.jpg}
        \caption{Original}
    \end{subfigure}
    \hfill
    \begin{subfigure}[b]{0.3\textwidth}
        \includegraphics[width=\textwidth]{figs/einsteinK_100.jpg}
        \caption{k = 100}
    \end{subfigure}
\end{figure}
\begin{figure}[h]
    \centering
    \begin{subfigure}[b]{0.3\textwidth}
        \includegraphics[width=\textwidth]{figs/einsteinK_10.jpg}
        \caption{k = 10}
    \end{subfigure}
    \hfill
    \begin{subfigure}[b]{0.3\textwidth}
        \includegraphics[width=\textwidth]{figs/einsteinK_20.jpg}
        \caption{k = 20}
    \end{subfigure}
    \hfill
    \begin{subfigure}[b]{0.3\textwidth}
        \includegraphics[width=\textwidth]{figs/einsteinK_50.jpg}
        \caption{k = 50}
    \end{subfigure}
    \caption{Effect of rank $k$ on compression quality.}
\end{figure}
For globe.jpg
\begin{figure}[h]
    \centering
    \begin{subfigure}[b]{0.3\textwidth}
        \includegraphics[width=\textwidth]{figs/globe.jpg}
        \caption{Original}
    \end{subfigure}
    \hfill
    \begin{subfigure}[b]{0.3\textwidth}
        \includegraphics[width=\textwidth]{figs/globeK_100.jpg}
        \caption{k = 100}
    \end{subfigure}
\end{figure}
\begin{figure}[h]
    \centering
    \begin{subfigure}[b]{0.3\textwidth}
        \includegraphics[width=\textwidth]{figs/globeK_10.jpg}
        \caption{k = 10}
    \end{subfigure}
    \hfill
    \begin{subfigure}[b]{0.3\textwidth}
        \includegraphics[width=\textwidth]{figs/globeK_20.jpg}
        \caption{k = 20}
    \end{subfigure}
    \hfill
    \begin{subfigure}[b]{0.3\textwidth}
        \includegraphics[width=\textwidth]{figs/globeK_50.jpg}
        \caption{k = 50}
    \end{subfigure}
    \caption{Effect of rank $k$ on compression quality.}
\end{figure}
\newpage
For greyscale.png
\begin{figure}[h]
    \centering
    \begin{subfigure}[b]{0.3\textwidth}
        \includegraphics[width=\textwidth]{figs/greyscale.png}
        \caption{Original}
    \end{subfigure}
    \hfill
    \begin{subfigure}[b]{0.3\textwidth}
        \includegraphics[width=\textwidth]{figs/greyscaleK_100.jpg}
        \caption{k = 100}
    \end{subfigure}
\end{figure}
\begin{figure}[h]
    \centering
    \begin{subfigure}[b]{0.3\textwidth}
        \includegraphics[width=\textwidth]{figs/greyscaleK_10.jpg}
        \caption{k = 10}
    \end{subfigure}
    \hfill
    \begin{subfigure}[b]{0.3\textwidth}
        \includegraphics[width=\textwidth]{figs/greyscaleK_20.jpg}
        \caption{k = 20}
    \end{subfigure}
    \hfill
    \begin{subfigure}[b]{0.3\textwidth}
        \includegraphics[width=\textwidth]{figs/greyscaleK_50.jpg}
        \caption{k = 50}
    \end{subfigure}
    \caption{Effect of rank $k$ on compression quality.}
\end{figure}
\newpage
\section{Error Analysis}
\begin{table}[h!]
    \centering
    \begin{tabular}{|c|c|}
    \hline
    \textbf{K}        & \textbf{Average Forbeanius Error} \\
    \hline
    10 \\    & $3249.14$ \\
    \hline
    20   \\      & $1568.472$ \\
    \hline
    50    \\         & $1048.647$ \\
    \hline
    100 \\   & $893.768$ \\
    \hline
    200 \\ & $567.423$ \\
    \hline
    \end{tabular}
    \caption{Error}
    \label{tab:error}
\end{table}
\section{Conclusion}
Image compression through power iteration is efficient and easy to implement and hence a good choice for image compression.
\end{document}